\documentclass[12pt,letterpaper]{article}
 % \usepackage[spanish]{babel} Si pones babel el $<$ mata el doc por alguna razon xd
\usepackage[utf8]{inputenc}
\usepackage[T1]{fontenc}
\usepackage{float}

\usepackage{xcolor} % Colores


\usepackage{semantic} % Inferencias

\usepackage{tikz} % Graficas
\usepackage{tikz-qtree} % Para arboles easy peasy \Tree
% Checar el archivo lambda-macros.sty
\usepackage{lambda-macros} % calculo lambda

%% Sets page size and margins
\usepackage[a4paper,top=2.5cm,bottom=2.5cm,left=2cm,right=2cm]{geometry}

\usepackage{amsmath}
\usepackage{amssymb}
\usepackage{amsthm}
\usepackage{mathtools}
\usepackage{listings}
\usepackage{enumitem}[shortlabels]
\usepackage[colorinlistoftodos]{todonotes}
\usepackage[colorlinks=true, allcolors=blue]{hyperref}
\usepackage{url}

\usepackage{lipsum}
\usepackage{bussproofs}
%Author affil
\usepackage{authblk}

%% Title
\title{
		\vspace{-0.7in} 	
		\usefont{OT1}{bch}{b}{n}
		
		\begin{minipage}{3cm}
        \vspace{-0.5in} 	
    	\begin{center}
    		%\includegraphics[height=3.2cm]{logo_unam.png}
    	\end{center}
    \end{minipage}\hfill
    \begin{minipage}{10.7cm}
    	\begin{center}

\normalfont \normalsize \textsc{UNIVERSIDAD NACIONAL AUTÓNOMA DE MÉXICO \\ FACULTAD DE CIENCIAS \\ LENGUAJES DE PROGRAMACIÓN } \\
		\huge Semanal 07

    	\end{center}
     
    \end{minipage}\hfill
    \begin{minipage}{3.2cm}
    \vspace{-0.5in} 
    	\begin{center}
    		%\includegraphics[height=3.2cm]{logo_fc.png}
    	\end{center}
    \end{minipage}

\author{Victor de Jesús Villalobos Ramírez - 313098675 \\ Miguel Angel Vargas Campos - 423114223}
\date{}

}
 
\begin{document}

\maketitle

\begin{enumerate}
    \item Dada la siguiente expresión en \textbf{MiniLisp}:
\begin{verbatim}
(let (sum (lambda (n) (if0 n 0 (+ n (sum (- n 1))))))
    (sum 5))
\end{verbatim}
    \begin{itemize}
        \item Ejecutarla y explicar el resultado.
        \begin{verbatim}
            Sustitución [sum:=(lambda (n) (if0 n 0 (+ n (sum (- n 1)))))]
            ((lambda (n) (if0 n 0 (+ n (sum (- n 1)))))5)
            Sustitución [n := 5]
            (if0 5 0 (+ 5 (sum (- 5 1)))) 
            (+ 5 (sum (- 5 1))) 
            (+ 5 (sum 4))
            error

        \end{verbatim}

         La ejecución genera un error de variable libre ya que no tiene manera de seguir, por que no sabe la definición de sum y así poder auto-referenciarse. \\ 
        
        \item Modificarla usando el combinador de punto fijo Y, volver a ejecutarla y explicar el resultado.
         \begin{verbatim}
             (let (sum (Y (lambda (sum) (lambda (n) (if0 n 0 (+ n (sum (- n 1))))))))
    (sum 5))

        Sustitución [sum:=(Y (lambda (sum) (lambda (n) (if0 n 0 (+ n (sum (- n 1)))))))]
        (Y (lambda (sum) (lambda (n) (if0 n 0 (+ n (sum (- n 1)))))5))
        Sustitución [n := 5]
        (Y (lambda (sum) (if0 5 0 (+ 5 (sum (- 5 1))))))
        (Y (lambda (sum) (+ 5(sum (- 5 1)))))
        (y (lambda (sum) (+ 5(sum 4))))
        

        Sustitución [sum:=(Y (lambda (sum) (lambda (n) (if0 n 0 (+ n (sum (- n 1)))))))]
        (Y (lambda (sum) (lambda (n) (if0 n 0 (+ n (sum (- n 1)))))4))
        Sustitución [n := 4]
        (Y (lambda (sum) (+ 5(if0 4 0 (+ 4 (sum (- 4 1)))))))
        (Y (lambda (sum) (+ 5(+ 4 (sum (- 4 1)))))
        (lambda (sum) (+ 5(+ 4 (sum 3)))Y)
        ...
        ...
        ...

        (+5 (+4 (+3 (+2 (+1 0)))))
        15
         \end{verbatim}

            El resultado final de la ejecución es 15 ya que gracias al usar el combinador de punto fijo Y podemos hacer auto-referencia a la función completa de sum y así evitar los problemas de variables libres que se podrían presentar, como en la ejecución anterior. \\
         
        
    \end{itemize}
    \item Evaluar la siguiente expresión en \textbf{Racket}, explicar su resultado y dar la continuación asociada a evaluar usando la notación $\lambda$↑
\begin{verbatim}
> (define c #f)
> (+ 1 (+ 2 (+ 3 (+ (let/cc k (set! c k) 4) 5))))
> (c 10)
\end{verbatim}
    \textbf{Evaluación:}
    \begin{figure}[H]
        \centering
        \includegraphics[width=0.5\linewidth]{racket.png}
        \caption{Ejecución del código en \textit{Racket}.}
        \label{fig:enter-label}
    \end{figure}
    \textbf{Explicación:}
    \begin{itemize}
        \item
            \begin{verbatim}
(define c #f)
            \end{verbatim}
        Definimos la variable $c$ como $\#f$.
        \item 
            \begin{verbatim}
(+ 1 (+ 2 (+ 3 (+ (let/cc k (set! c k) 4) 5))))
            \end{verbatim}
        Se ejecuta:
        \begin{verbatim}
(+ 1 (+ 2 (+ 3 (+ 4 5))))
        \end{verbatim}
        Para obtener $15$ y guardamos en $c$ la continuación asociada al punto donde está el $4$.
        \item 
            \begin{verbatim}
(c 10)
            \end{verbatim}
        Ejecutamos la continuación que habíamos almacenado en $c$ pasando $10$ como parámetro real y obtenemos $21$ como resultado.
    \end{itemize}
    \textbf{Continuación asociada:}
    \\ \\
    $\lambda$↑$v.(+\ 1\ (+\ 2\ (+\ 3\ (+\ v \ 5))))$
    \item Realizar los siguientes ejercicios en \textbf{Haskell}:
    \\ \\
    Definir la función recursiva \textit{ocurrenciasElementos} que toma como argumentos dos listas y devuelve una lista de parejas, en donde cada pareja contiene en su parte izquierda un elemento de la segunda lista y en su parte derecha el número de veces que aparece dicho elemento en la primera lista. Por ejemplo:
\begin{verbatim}
> ocurrenciasElementos [1,3,6,2,4,7,3,9,7] [5,2,3]
[(5,0),(2,1),(3,2)]
\end{verbatim}

Funcion en \textbf{Haskell}:

\begin{verbatim}
ocurrenciasElementos :: Eq a => [a] -> [a] -> [(a,Integer)]
ocurrenciasElementos _ [] = []
ocurrenciasElementos list (y:ys)
    = (y, count list y) : ocurrenciasElementos list ys

count :: Eq a => [a] -> a -> Integer
count [] _ = 0
count (x:xs) elem
    | x == elem = 1 + count xs elem
    | otherwise = count xs elem
    
\end{verbatim}

    \begin{itemize}
        \item Mostrar los registros de activación generados por la función definida en el ejercicio anterior con la llamada \textit{ocurrenciasElementos [1,2,3] [1,2]}.
\begin{verbatim}
> ocurrenciasElementos [1,2,3] [1,2]
> (1, count [1,2,3] 1) : ocurrenciasElementos [1,2,3] [2]
> (1, 1 + count [2,3] 1) : ocurrenciasElementos [1,2,3] [2]
> (1, 1 + count [3] 1) : ocurrenciasElementos [1,2,3] [2]
> (1, 1 + count [] 1) : ocurrenciasElementos [1,2,3] [2]
> (1, 1 + 0) : ocurrenciasElementos [1,2,3] [2]
> (1, 1) : ocurrenciasElementos [1,2,3] [2]
> (1, 1) : (2, count [1,2,3] 2) : ocurrenciasElementos [1,2,3] []
> (1, 1) : (2, count [2,3] 2) : ocurrenciasElementos [1,2,3] []
> (1, 1) : (2, 1 + count [3] 2) : ocurrenciasElementos [1,2,3] []
> (1, 1) : (2, 1 + count [] 2) : ocurrenciasElementos [1,2,3] []
> (1, 1) : (2, 1 + 0) : ocurrenciasElementos [1,2,3] []
> (1, 1) : (2, 1) : ocurrenciasElementos [1,2,3] []
> (1, 1) : (2, 1) : []
\end{verbatim}
        \item Optimizar la función definida usando recursión de cola. Deben transformar todas las funciones auxiliares que utilicen.
\begin{verbatim}
ocurrenciasElementos :: Eq a => [a] -> [a] -> [(a,Integer)]
ocurrenciasElementos list search = aux [] list search

aux :: Eq a => [(a,Integer)] -> [a] -> [a] -> [(a,Integer)]
aux acc _ [] = acc
aux acc list (y:ys)
    = aux ((y, count 0 list y) : acc) list ys

count :: Eq a => Integer -> [a] -> a -> Integer
count acc [] _ = acc
count acc (x:xs) elem
    | x == elem = count (acc + 1) xs elem
    | otherwise = count acc xs elem
\end{verbatim}
        \item Mostrar los registros de activación generados por la versión de cola con la misma llamada.
\begin{verbatim}
> ocurrenciasElementos [1,2,3] [1,2]
> aux [] [1,2,3] [1,2]
> aux ((1, count 0 [1,2,3] 1) : []) [1,2,3] [2]
> aux ((1, count 1 [2,3] 1) : []) [1,2,3] [2]
> aux ((1, count 1 [3] 1) : []) [1,2,3] [2]
> aux ((1, count 1 [] 1) : []) [1,2,3] [2]
> aux ((1, 1) : []) [1,2,3] [2]
> aux ((2, count 0 [1,2,3] 2) : (1, 1) : []) [1,2,3] []
> aux ((2, count 0 [2,3] 2) : (1, 1) : []) [1,2,3] []
> aux ((2, count 1 [3] 2) : (1, 1) : []) [1,2,3] []
> aux ((2, count 1 [] 2) : (1, 1) : []) [1,2,3] []
> aux ((2, 1) : (1, 1) : []) [1,2,3] []
> (2, 1) : (1, 1) : []
\end{verbatim}
    \end{itemize}
\end{enumerate}

\end{document}